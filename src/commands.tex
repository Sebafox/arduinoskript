% Befehle mit tcolorbox-Paket
\newcommand{\button}[1]{\tcbox[size=fbox,box align = base,colback=gray!20, colframe=gray,nobeforeafter]{\texttt{#1}}}
%\newtcolorbox{zsfg}{before=\medskip, enhanced, interior style={top color=CadetBlue!80!green, bottom color=DarkCyan!70!DarkGreen}, coltext=white, colframe=white, after=\medskip}
%\newtcolorbox{zsfg}[1]{before=\medskip, breakable, enhanced, colback=CadetBlue!70!green, coltext=black, colframe=DarkCyan!70!DarkGreen, fonttitle=\bfseries, after=\medskip, title=#1, attach title to upper, after title={\medskip}}
\newtcolorbox{zsfg}[1]{before=\medskip, breakable, enhanced, colback=CadetBlue!70!green, coltext=black, frame style={left color=CadetBlue!75!green, right color=DarkCyan!70!DarkGreen},fonttitle=\bfseries, after=\medskip, title=#1}
\newtcolorbox{ziel}{before=\medskip, breakable, sharp corners=all, enhanced, frame style={left color=CadetBlue!70!green, right color=DarkCyan!70!DarkGreen}, after=\medskip}


%	Definition der Aufgabenumgebung
\newcounter{aufgabennummer}[chapter]	%durch das optionale Argument wird der section-Counter als übergeordneter Zähler festgelegt,
% sodass die aufgabennummer auf null gesetzt wird, wenn section erhöht wird
\newenvironment{aufgabe}{%
	\medbreak %
	\textbf{\sffamily Aufgabe}%
	\stepcounter{aufgabennummer}%Counter beginnt immer bei 0, also vor der Ausgabe hochzählen
%	\ifnum \value{section}>0 %
%		~\textbf{\sffamily\thesection.\theaufgabennummer}%
%	\else ~\textbf{\sffamily\theaufgabennummer}%
%	\fi% falls die Aufgabe in einem Abschnitt steht, wird dieser zur Nummerierung herangezogen
	~\textbf{\sffamily\theaufgabennummer}:~%
}{ %
	\par\medskip %

}

\newenvironment{aufgabe*}{%
	\medbreak %
	\textbf{\sffamily Aufgabe}%
	:~%
}{ %
	\par\medskip %
	
}

\newcounter{projektnummer}[chapter]
\newenvironment{projekt}[1][\unskip]{%
	\begin{tcolorbox}[breakable,before=\medbreak, enhanced, frame hidden, interior hidden, borderline west={1mm}{-4mm}{CadetBlue!70!green}, top=0mm,bottom=0mm,boxsep=0mm, left=0mm, right=0mm, after=\medskip]
	\textbf{\sffamily Projekt}%
	\stepcounter{projektnummer}% 
%	\ifnum \value{section}>0 %
%	~\textbf{\sffamily\thesection.\theprojektnummer}%
%	\else ~\textbf{\sffamily\theprojektnummer}%
%	\fi% falls die Aufgabe in einem Abschnitt steht, wird dieser zur Nummerierung herangezogen
	~\textbf{\sffamily\theprojektnummer}:~%
	\textbf{#1} \smallbreak%
}{ %
	\end{tcolorbox}
	\par\medskip %
	 %
}

\newenvironment{projekt*}[1][\unskip]{%
	\begin{tcolorbox}[breakable,before=\medbreak, enhanced, frame hidden, interior hidden, borderline west={1mm}{-4mm}{CadetBlue!70!green}, top=0mm,bottom=0mm,boxsep=0mm, left=0mm, right=0mm, after=\medskip]
		\textbf{\sffamily Projekt}%
		:~%
		\textbf{#1} \smallbreak%
	}{ %
	\end{tcolorbox}
	\par\medskip %
	%
}

\newenvironment{links}{%
	\begin{tcolorbox}[before=\bigbreak, enhanced, frame hidden, colback=gray!20, after=\medskip]
	\wlan \textbf{\sffamily Motivationsquellen}%
	\begin{itemize}[noitemsep, itemindent=0pt, leftmargin=5mm, label=>]
}{%
	\end{itemize}
	\end{tcolorbox}
	\par\medskip %
	%
}

\newenvironment{projektueberblick}{%
	\begin{tcolorbox}[before=\bigbreak, enhanced, frame hidden, colback=DarkCyan!70!DarkGreen, coltext=gray!30, after=\medskip]
	\textbf{\sffamily Projekte in diesem Kapitel:}%
	\setlength{\columnsep}{2cm}
	\begin{multicols}{2}
	\begin{itemize}[noitemsep, itemindent=0pt, leftmargin=5mm, label=>]
	}{%
	\end{itemize}
	\end{multicols}
	\end{tcolorbox}
	\par\medskip %
	%
}

\newenvironment{recherche}[1]{%
	\medbreak
	\lupe ~ \textbf{Recherche:~#1}
	\par
}{%
	\par \medskip
}

\definecolor{blueviolet}{RGB}{138,43,226}
\definecolor{turquoise1}{RGB}{0,245,255}
\definecolor{chocolate}{RGB}{210,105,30}
\definecolor{olivedrab}{RGB}{107,142,35}
\definecolor{khaki}{RGB}{240,230,140}
\definecolor{deeppink}{RGB}{255,20,147}
\definecolor{navyblue}{RGB}{0,0,128}
\definecolor{aquamarine}{RGB}{127,255,212}
\definecolor{sandybrown}{RGB}{244,164,96}
\definecolor{seagreen}{RGB}{46,139,87}
\definecolor{coral}{RGB}{255,127,80}
\definecolor{darkorchid}{RGB}{153,50,204}
\definecolor{seagreen3}{RGB}{67,205,128}
\definecolor{dodgerblue}{RGB}{30,144,255}
\definecolor{burlywood4}{RGB}{139,115,85}


\definecolor{nepoLogik}{RGB}{51,184,202}

%\newenvironment{wichtig}[1][]{%
%	\begin{mdframed}[%
%		backgroundcolor={gray!15}, hidealllines=true,
%		skipabove=0.7\baselineskip, skipbelow=0.7\baselineskip,
%		splitbottomskip=2pt, splittopskip=4pt, #1]%
%		\makebox[0pt]{
%			\smash{\raisebox{0pt}[0pt][0pt]{%
%					\fontsize{30pt}{30pt}\selectfont%
%					\hspace*{-23pt}%
%					\raisebox{-2pt}{%
%						{\color{gray!80!black}\sffamily\bfseries !}%
%					}%
%				}}%
%			}%
%		}{\end{mdframed}}
%	
%\newenvironment{info}[1][]{%
%	\begin{mdframed}[%
%		skipabove=0.7\baselineskip, skipbelow=0.7\baselineskip, roundcorner=5pt, hidealllines=true,%
%		frametitlefont={\bfseries\sffamily\normalsize\color{white}}, backgroundcolor=gray!10, frametitlebackgroundcolor=black!70!blue,%
%		frametitleaboveskip=6pt, frametitlebelowskip=6pt, innertopmargin=8pt, #1]%
%		\makebox[0pt]{
%			\smash{\raisebox{0pt}[0pt][0pt]{%
%					\fontsize{30pt}{30pt}\selectfont%
%					\hspace*{0pt}%
%					\raisebox{12pt}{%
%						{\infosym}%
%					}%
%				}}%
%			}%
%		}{\end{mdframed}}
%	
%% grau hinterlegte Randbemerkung definieren
%\newcommand{\randnotiz}[1]{\marginpar{\vspace{-0.84\baselineskip}%mdframed fügt automatisch etwas vertikalen Abstand ein, der hiermit wieder weg gemacht wird und zwar so, dass der Text unten bündig ist (aus dem Verhältnis footnotesize zu normalsize ergibt sich 0.84)
%\begin{mdframed}[backgroundcolor=gray!10, hidealllines=true, innerrightmargin=3pt, innerleftmargin=3pt] %
%	\footnotesize#1%
%\end{mdframed} %
%}}

%\newenvironment{code}{%
%	\begin{center}
%	\begin{minipage}[t]{0.9\textwidth}
%	\begin{lstlisting}[frame=single, framerule=0pt, style=Arduino]
%}{%
%	\end{lstlisting}
%	\end{minipage}
%	\end{center}
%}

%Vorlage 1:
%\begin{figure}[h]
%   \centering
%   \includegraphics[width=0.8\textwidth]{xx.eps}
%   \caption[Kurz für Abbildungsverzeichnis]{Lang}
%   \label{abb:}
%\end{figure}

%Vorlage 2:
%\begin{wrapfigure}{r}[Überhang in Rand]{0.4\textwidth}
%   \centering
%   \includegraphics{xx.eps}
%   \setcapindent{24pt}
%   \caption{}
%   \label{abb:}
%\end{wrapfigure}

% Vorlage 3:
%\begin{figure}[H]
%	\centering
%	\subfloat[Kurzbeschreibung]{\includegraphics[width=0.49\textwidth]{Leistungskurve_TEM00_gross.png}}
%	\subfloat[Kurzbeschreibung]{\includegraphics[width=0.49\textwidth]{Strahl-undPumpleistung_gross.png}}
%	\caption[Leistungskurve des fertigen Lasers]{}
%	\label{abb:leistungtem00}
%\end{figure}

% Vorlage:
%\begin{table}[h]
%   \centering
%   \begin{minipage}[c]{\textwidth}
%      \begin{tabu} to \textwidth {X[C]X[L,$$]}
%         \toprule
%         Spalte 1 & Spalte 2 \\
%         \midrule
%         Spalte 1 & Spalte 2 \\
%         Spalte 1 & Spalte 2 \\
%         \bottomrule
%      \end{tabu}
%   \end{minipage}
%   \caption{}
%   \label{tab:}
%\end{table}